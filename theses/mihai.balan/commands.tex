% Creates the text for referring to the figure identified by the key given as
% the first argument.
\newcommand{\figname}[1]{\emph{figura \ref{#1}}}

% Inserts a new figure, with these parameters:
%	#1 - filename of the figure to be inserted
%	#2 - caption for the figure
%	#3 - label for the figure (used in \ref-ing the figure)
\newcommand{\insfig}[3]{
	\begin{figure}[!htb]
		\begin{center}
			\fbox{
				%\includegraphics[width=0.75\textwidth]{#1}
 				\includegraphics{#1}
			}
			\caption{#2\label{#3}}
			
	    \end{center}
	\end{figure}
}

% Inserts a new figure, with these parameters:
%	#1 - filename of the figure to be inserted
%	#2 - caption for the figure
%	#3 - label for the figure (used in \ref-ing the figure)
%	#4 - fraction of the text width occupied by the figure
\newcommand{\insfigw}[4]{
	\begin{figure}[!htb]
		\begin{center}
			\fbox{
				\includegraphics[width=#4\textwidth]{#1}
			}
			\caption{#2\label{#3}}
			
	    \end{center}
	\end{figure}
}

% Inserts a new figure, with these parameters:
%	#1 - filename of the figure to be inserted
%	#2 - caption for the figure
%	#3 - caption for display in the list of figures
%	#4 - label for the figure (used in \ref-ing the figure)
%	#5 - fraction of the text width occupied by the figure
\newcommand{\insfigshw}[5]{
	\begin{figure}[!htb]
		\begin{center}
			\fbox{
				\includegraphics[width=#5\textwidth]{#1}
			}
			\caption[#3]{#2\label{#4}}
			
	    \end{center}
	\end{figure}
}